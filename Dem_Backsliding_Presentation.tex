\RequirePackage{atbegshi}
\documentclass[compress,aspectratio=169]{beamer}\usepackage[]{graphicx}\usepackage[]{xcolor}
% maxwidth is the original width if it is less than linewidth
% otherwise use linewidth (to make sure the graphics do not exceed the margin)
\makeatletter
\def\maxwidth{ %
  \ifdim\Gin@nat@width>\linewidth
    \linewidth
  \else
    \Gin@nat@width
  \fi
}
\makeatother

\definecolor{fgcolor}{rgb}{0.345, 0.345, 0.345}
\newcommand{\hlnum}[1]{\textcolor[rgb]{0.686,0.059,0.569}{#1}}%
\newcommand{\hlstr}[1]{\textcolor[rgb]{0.192,0.494,0.8}{#1}}%
\newcommand{\hlcom}[1]{\textcolor[rgb]{0.678,0.584,0.686}{\textit{#1}}}%
\newcommand{\hlopt}[1]{\textcolor[rgb]{0,0,0}{#1}}%
\newcommand{\hlstd}[1]{\textcolor[rgb]{0.345,0.345,0.345}{#1}}%
\newcommand{\hlkwa}[1]{\textcolor[rgb]{0.161,0.373,0.58}{\textbf{#1}}}%
\newcommand{\hlkwb}[1]{\textcolor[rgb]{0.69,0.353,0.396}{#1}}%
\newcommand{\hlkwc}[1]{\textcolor[rgb]{0.333,0.667,0.333}{#1}}%
\newcommand{\hlkwd}[1]{\textcolor[rgb]{0.737,0.353,0.396}{\textbf{#1}}}%
\let\hlipl\hlkwb

\usepackage{framed}
\makeatletter
\newenvironment{kframe}{%
 \def\at@end@of@kframe{}%
 \ifinner\ifhmode%
  \def\at@end@of@kframe{\end{minipage}}%
  \begin{minipage}{\columnwidth}%
 \fi\fi%
 \def\FrameCommand##1{\hskip\@totalleftmargin \hskip-\fboxsep
 \colorbox{shadecolor}{##1}\hskip-\fboxsep
     % There is no \\@totalrightmargin, so:
     \hskip-\linewidth \hskip-\@totalleftmargin \hskip\columnwidth}%
 \MakeFramed {\advance\hsize-\width
   \@totalleftmargin\z@ \linewidth\hsize
   \@setminipage}}%
 {\par\unskip\endMakeFramed%
 \at@end@of@kframe}
\makeatother

\definecolor{shadecolor}{rgb}{.97, .97, .97}
\definecolor{messagecolor}{rgb}{0, 0, 0}
\definecolor{warningcolor}{rgb}{1, 0, 1}
\definecolor{errorcolor}{rgb}{1, 0, 0}
\newenvironment{knitrout}{}{} % an empty environment to be redefined in TeX

\usepackage{alltt} % aspectratio=169


% % % % % % % % % % % % % % %
%             MY PACKAGES 
% % % % % % % % % % % % % % %
\usepackage{graphicx}       % Use pdf, png, jpg, or eps with pdflatex; use eps in DVI mode
\usepackage{dcolumn} % this pack is neccesary to build nicer columns with texreg--dont remove it.
\usepackage[export]{adjustbox}
\usepackage{xcolor}[dvipsnames]
\usepackage{amssymb,amsmath}
\usepackage{threeparttable} % package to have long notes in reg tables in texreg. 
\usepackage{graphics}
\usepackage{pgfplots}
\pgfplotsset{compat=1.11}
\usepgfplotslibrary{fillbetween}
\usepackage{fontawesome}


%\usepackage{tipx}
%\usepackage{tikz}
%\usetikzlibrary{arrows,shapes,decorations.pathmorphing,backgrounds,positioning,fit,petri}
\usepackage{rotating}
%\usepackage{scalerel} % for inline images
\usepackage{import}
%\usepackage{times}
\usepackage{array}
\usepackage{tabularx}
\usepackage{booktabs}
%\usepackage{textcomp}
\usepackage{float}
%\usepackage{setspace}      % \doublespacing \singlespacing \onehalfspacing %doble espacio
%\label{x:y}                          %ocupar para autoref.
%\autoref{x:y}                        %ocupar para autoref.
%\usepackage{nopageno}      %desactivar para p�ginas
\usepackage{pifont}
\newcommand{\xmark}{\ding{55}}%

%\usepackage{marvosym} %faces

\usepackage{hyperref}
\hypersetup{
    %bookmarks=true,         % show bookmarks bar?
    unicode=false,          % non-Latin characters in Acrobat’s bookmarks
    pdftoolbar=true,        % show Acrobat’s toolbar?
    pdfmenubar=true,        % show Acrobat’s menu?
    pdffitwindow=true,     % window fit to page when opened
    pdfstartview={FitH},    % fits the width of the page to the window
    pdftitle={My title},    % title
    pdfauthor={Author},     % author
    pdfsubject={Subject},   % subject of the document
    pdfcreator={Creator},   % creator of the document
    pdfproducer={Producer}, % producer of the document
    pdfkeywords={keyword1} {key2} {key3}, % list of keywords
    pdfnewwindow=true,      % links in new window
    colorlinks=true,       % false: boxed links; true: colored links
    linkcolor=blue,          % color of internal links (change box color with linkbordercolor)
    citecolor=blue,        % color of links to bibliography
    filecolor=blue,      % color of file links
    urlcolor=blue           % color of external links
}

\hypersetup{
  colorlinks = true,
  urlcolor = blue,
  pdfpagemode = UseNone
}


\usepackage{multirow}

\usepackage{tikz}
\usetikzlibrary{arrows,decorations.pathreplacing}



\usepackage{listings}
\usepackage{color}
\definecolor{dkgreen}{rgb}{0,0.6,0}
\definecolor{gray}{rgb}{0.5,0.5,0.5}
\definecolor{mauve}{rgb}{0.58,0,0.82}
\lstset{ %
  language=R,                     % the language of the code
  basicstyle=\TINY,           % the size of the fonts that are used for the code
  numbers=left,                   % where to put the line-numbers
  numberstyle=\tiny\color{gray},  % the style that is used for the line-numbers
  stepnumber=1,                   % the step between two line-numbers. If it's 1, each line
                                  % will be numbered
  numbersep=5pt,                  % how far the line-numbers are from the code
  backgroundcolor=\color{white},  % choose the background color. You must add \usepackage{color}
  showspaces=false,               % show spaces adding particular underscores
  showstringspaces=false,         % underline spaces within strings
  showtabs=false,                 % show tabs within strings adding particular underscores
  frame=single,                   % adds a frame around the code
  rulecolor=\color{black},        % if not set, the frame-color may be changed on line-breaks within not-black text (e.g. commens (green here))
  tabsize=1,                      % sets default tabsize to 2 spaces
  captionpos=b,                   % sets the caption-position to bottom
  breaklines=true,                % sets automatic line breaking
  breakatwhitespace=false,        % sets if automatic breaks should only happen at whitespace
  title=\lstname,                 % show the filename of files included with \lstinputlisting;
                                  % also try caption instead of title
  keywordstyle=\color{blue},      % keyword style
  commentstyle=\color{dkgreen},   % comment style
  stringstyle=\color{mauve},      % string literal style
  escapeinside={\%*}{*)},         % if you want to add a comment within your code
  morekeywords={*,...}            % if you want to add more keywords to the set
} 

% % % % % % % % % % % % % % %
%           PACKAGE CUSTOMIZATION
% % % % % % % % % % % % % % %

% GENERAL CUSTOMIZATION
\usepackage[math]{iwona}% font
\usetheme{Singapore}  % template I should use
%\usetheme{Szeged}  % alternative template
\usecolortheme{rose}  % color template
\makeatletter     % to show subsection/section title (1/3)
\beamer@theme@subsectiontrue % to show subsection/section title (2/3)
\makeatother      % to show subsection/section title (3/3)



% THIS BELOW IS TO MAKE NAVIGATION DOTS MARKED DURING PRESENTATION
\makeatletter
\def\slideentry#1#2#3#4#5#6{%
  %section number, subsection number, slide number, first/last frame, page number, part number
  \ifnum#6=\c@part\ifnum#2>0\ifnum#3>0%
    \ifbeamer@compress%
      \advance\beamer@xpos by1\relax%
    \else%
      \beamer@xpos=#3\relax%
      \beamer@ypos=#2\relax%
    \fi%
  \hbox to 0pt{%
    \beamer@tempdim=-\beamer@vboxoffset%
    \advance\beamer@tempdim by-\beamer@boxsize%
    \multiply\beamer@tempdim by\beamer@ypos%
    \advance\beamer@tempdim by -.05cm%
    \raise\beamer@tempdim\hbox{%
      \beamer@tempdim=\beamer@boxsize%
      \multiply\beamer@tempdim by\beamer@xpos%
      \advance\beamer@tempdim by -\beamer@boxsize%
      \advance\beamer@tempdim by 1pt%
      \kern\beamer@tempdim
      \global\beamer@section@min@dim\beamer@tempdim
      \hbox{\beamer@link(#4){%
          \usebeamerfont{mini frame}%
          \ifnum\c@section>#1%
            %\usebeamercolor[fg]{mini frame}%
            %\usebeamertemplate{mini frame}%
            \usebeamercolor{mini frame}%
            \usebeamertemplate{mini frame in other subsection}%
          \else%
            \ifnum\c@section=#1%
              \ifnum\c@subsection>#2%
                \usebeamercolor[fg]{mini frame}%
                \usebeamertemplate{mini frame}%
              \else%
                \ifnum\c@subsection=#2%
                  \usebeamercolor[fg]{mini frame}%
                  \ifnum\c@subsectionslide<#3%
                    \usebeamertemplate{mini frame in current subsection}%
                  \else%
                    \usebeamertemplate{mini frame}%
                  \fi%
                \else%
                  \usebeamercolor{mini frame}%
                  \usebeamertemplate{mini frame in other subsection}%
                \fi%
              \fi%
            \else%
              \usebeamercolor{mini frame}%
              \usebeamertemplate{mini frame in other subsection}%
            \fi%
          \fi%
        }}}\hskip-10cm plus 1fil%
  }\fi\fi%
  \else%
  \fakeslideentry{#1}{#2}{#3}{#4}{#5}{#6}%
  \fi\ignorespaces
  }
\makeatother


%%% bib begin
\usepackage[authordate,isbn=false,doi=false,url=false,eprint=false]{biblatex-chicago}
\DeclareFieldFormat[article]{title}{\mkbibquote{#1}} % make article titles in quotes
\DeclareFieldFormat[thesis]{title}{\mkbibemph{#1}} % make theses italics

\AtEveryBibitem{\clearfield{month}}
\AtEveryCitekey{\clearfield{month}}

\addbibresource{/Users/hectorbahamonde/Bibliografia_PoliSci/library.bib} 


% USAGES
%% use \textcite to cite normal
%% \parencite to cite in parentheses
%% \footcite to cite in footnote
%% the default can be modified in autocite=FOO, footnote, for ex. 
%%% bib end



% % % % % % % % % % % % % % %
%       To show the TITLE at the Bottom of each slide
% % % % % % % % % % % % % % %

\beamertemplatenavigationsymbolsempty 
\makeatletter
\setbeamertemplate{footline}
{
\leavevmode%
\hbox{%
\begin{beamercolorbox}[wd=1\paperwidth,ht=2.25ex,dp=2ex,center]{title in head/foot}%
\usebeamerfont{title in head/foot}\insertshorttitle
\end{beamercolorbox}%
\begin{beamercolorbox}[wd=1
\paperwidth,ht=2.25ex,dp=2ex,center]{date in head/foot}%
\end{beamercolorbox}}%
}
\makeatother



% to switch off navigation bullets
%% using \miniframeson or \miniframesoff
\makeatletter
\let\beamer@writeslidentry@miniframeson=\beamer@writeslidentry
\def\beamer@writeslidentry@miniframesoff{%
  \expandafter\beamer@ifempty\expandafter{\beamer@framestartpage}{}% does not happen normally
  {%else
    % removed \addtocontents commands
    \clearpage\beamer@notesactions%
  }
}
\newcommand*{\miniframeson}{\let\beamer@writeslidentry=\beamer@writeslidentry@miniframeson}
\newcommand*{\miniframesoff}{\let\beamer@writeslidentry=\beamer@writeslidentry@miniframesoff}
\makeatother

% Image full size: use 
%%\begin{frame}
  %%\fullsizegraphic{monogram.jpg}
%%\end{frame}
\newcommand<>{\fullsizegraphic}[1]{
  \begin{textblock*}{0cm}(-1cm,-3.78cm)
  \includegraphics[width=\paperwidth]{#1}
  \end{textblock*}
}


% hyperlinks
\hypersetup{colorlinks,
            urlcolor=[rgb]{0.01, 0.28, 1.0},
            linkcolor=[rgb]{0.01, 0.28, 1.0}}



%\newcommand{\vitem}[]{\vfill \item}

% % % % % % % % % % % % % % %
%           DOCUMENT ID
% % % % % % % % % % % % % % %

\title{\input{title.txt}\unskip} % 




\author[shortname]{
Hector Bahamonde \inst{1} \and 
Inga Saikkonen \inst{2} \and 
Mart Trasberg \inst{3} \\ \vspace{3mm} \scriptsize{\color{gray}Authors in alphabetical order. All contributed equally to this project.}
}

\institute[shortinst]{\inst{1} University of Turku, Finland \and %
                      \inst{2} \textnormal{Åbo Akademi}, Finland \and
                      \inst{3} Monterrey Tec, Mexico}







\date{\today}

%to to see shadows of previous blocks
%\setbeamercovered{dynamic}
\IfFileExists{upquote.sty}{\usepackage{upquote}}{}
\begin{document}















% % % % % % % % % % % % % % %
%           CONTENT
% % % % % % % % % % % % % % %

%% title frame
\begin{frame}
\titlepage
\end{frame}


\section{Introduction}


\subsection{Motivation}

\miniframeson
\begin{frame}[c]{Democratic Backsliding}
  \begin{itemize}
    \item Parece existir un consenso en que algunas democracias est\'an en riesgo de retroceder, acerc\'andose m\'as a sistemas autoritarios.

    \item Estos retrocesos han sido estudiados en un sinn\'umero de casos.

        \begin{itemize}
          \item \textcite{Kaufman2019} explican que ``a transition to competitive authoritarianism in the United States is unlikely, although not impossible.''

          \item {\bf {\color{red}Caso 2}.}
          \item {\bf {\color{red}Caso 3}.}

        \end{itemize}

  \end{itemize}
\end{frame} 

\miniframesoff
\begin{frame}[c]{Democratic Backsliding: A ``Winners Bias''}

Desafortunadamente, la major\'ia ha concentrado sus esfuerzos en c\'omo el {\bf ejecutivo} \emph{agranda} sus poderes. 

    \begin{itemize}
      \item \textcite[p. 2]{Perez-Linan2018} explica que ``most threats to democracy originate in the {\bf \emph{executive}}, not in congress.''

      \item ``Democratic backsliding is the incremental erosion of institutions [...] that results from the actions of [...] {\bf \emph{elected} governments} \parencite[p. 27]{Haggard2021}.''

      \item \textcite[p. 41]{Corrales2020} explica que ``electoral irregularities contributed to democratic backsliding in Venezuela under {\bf \emph{chavista} rule}.''

    \end{itemize}

\begin{alertblock}{What about the lossers?}
Qu\'e ocurre con los que pierden la elecci\'on? Existen diferencias sistem\'aticas en cuanto la tolerancia de acciones no democr\'aticas entre ``ganadores'' y ``perdedores''?
\end{alertblock}
    
\end{frame}

\subsection{Our Paper}

\miniframeson
\begin{frame}[c]{Nuestro Paper}

  \begin{itemize}
    \item A diferencia de la mayoría de las investigaciones que se concentran en posibles violaciones de los valores democr\'aticos por parte de los ``ganadores,'' nosotros dirigimos nuestra atenci\'on hacia los ``{\bf perdedores}'' electorales.

    \item Hicimos un survey experiment (pre-registrado) en dos democracias recientes, Chile (y Estonia).

    \item {\color{blue} Entender si los votantes que apoyaron al {\bf candidato perdedor} est\'an más abiertos a respaldar acciones anti-sist\'emicas {\bf contra el incumbente}}.

    \item Para esto, incluimos una teor\'ia enfocada en \emph{pérdidas} y \emph{loss aversion} (\emph{prospect theory}, e.g., \cite{Kahneman1979}).

    %\item {\bf Pre-registered findings}: encontramos que los votantes de {\bf Kast} \emph{no} son mas proclives que los votantes de {\bf Boric} a apoyar acciones antisist\'emicas (protestas). 

  \end{itemize}

      \begin{exampleblock}{Pre-registered findings} 
      Encontramos que los votantes de {\bf Kast} \emph{no} son mas proclives que los votantes de {\bf Boric} a apoyar acciones antisist\'emicas (protestas) que pongan en peligro el status quo. 
      \end{exampleblock}
  
\end{frame}


\section{Theory}

\subsection{Democratic Backsliding}

\miniframeson
\begin{frame}[c]{Democratic Backsliding}

  \begin{itemize}
    \item Test
  \end{itemize}

\end{frame}

\subsection{Prospect Theory}

\miniframeson
\begin{frame}[c]{Prospect Theory}

  \begin{itemize}
    \item Test
  \end{itemize}

\end{frame}


\section{Argument}

\subsection{Argument}


\miniframeson
\begin{frame}


\begin{alertblock}{Argument}
Test.
\end{alertblock}

%\begin{exampleblock}{}%In simple...
%\end{exampleblock}


\end{frame}


\miniframeson
\begin{frame}<presentation:0>%{Argument}
Test. %\centering\includegraphics[scale=1.2]{argument.pdf}
\end{frame}

\section{Empirics}

\subsection{Case}

\miniframeson
\begin{frame}[c]{Chilean Case}
  \begin{itemize}
    \item We follow a ``{\bf least-likely case design}'' {\color{gray}\tiny(Levy 2008).} Finland has been consistently considered as:
      \begin{itemize}
        \item[-] A `democratic' {\color{gray}\tiny (Polity-V)}.
        \item[-] An `economic egalitarian' {\color{gray}\tiny (Waltl 2022)}.
        \item[-] A `gender egalitarian.'
        \item[-] A `social-mobility prone' country {\color{gray}\tiny (Erola 2009)}.
        %\item[-] Having low-information Municipal Elections {\color{gray}\tiny(Berggren et al. 2010 and 2017)}.
      \end{itemize}
      \item Thus, it should be {\bf hard to find} any {\bf correlation} between {\color{blue}class-congruent use of status symbols} and {\bf voting}.
    \end{itemize}
\vspace{0.5cm}
...and yet, we \emph{do}.
\end{frame}




\subsection{Statistical Analyses}


\miniframeson
\begin{frame}[c]{\hypertarget{reg:table:slide}{Functional Form and Model}}

 \begin{align*}
 Y_{i}=\text{Votes}_{i}\sim\;&\text{\texttt{Poisson}}\\
 \text{log(Votes}_{i})  = \;
 & \beta_{1}\text{{\only<2,12>{\color{red}}Occupation-Appearance Congruence}}_{i}\times\text{{\only<2,12>{\color{red}}Social Class}}_{i} + \\
 & \beta_{2}\text{{\only<3>{\color{red}}Age}}_{i} + \\
 & \gamma{1}\text{{\only<4>{\color{blue}}Party}}_{i} + \\
 & \gamma{2}\text{{\only<5>{\color{blue}}City}}_{i} + \\
 & \mathbf{\Theta}_{i}
 \end{align*}
 \begin{itemize}
  \item In  $\mathbf{\Theta}$ we {\bf also control for}: ${\only<6>{\color{red}}\text{Attractiveness}_{i}}$, ${\only<7>{\color{red}}\text{Masculinity}_{i}}$ and ${\only<8>{\color{red}}\text{Femininity}_{i}}$.
 \item {\only<9>{\color{red}}Full}, but also {\bf partition the data} ({\only<10>{\color{red}}male} \& {\only<11>{\color{red}}female}).
 \item We focus on the {\only<12>{\color{red}}{\bf marginal effects}}  of the {\only<12>{\color{red}}interaction term}.
 \\
 \hspace{10.7cm} \hyperlink{reg:table}{\beamergotobutton{show regression table}}
\end{itemize}
\end{frame}





\section{Discussion}

\subsection{At a Glance}

\miniframeson
\begin{frame}[c]{Main Results}
test
\end{frame}

\subsection{Wrapping Up}

\miniframeson
\begin{frame}[c]{Main Takeaways}
  \begin{itemize}
    \item[$\checkmark$] Test.
  \end{itemize}
\end{frame}

\miniframesoff
\begin{frame}[c]{Thank you}
        \begin{center}
          \vspace{-0.7cm}\includegraphics[scale=.06, center]{/Users/hectorbahamonde/hbahamonde.github.io/resources/qr-code.pdf}
        \end{center}


        \begin{itemize}
            %\item Paper (draft) available at {\color{blue}www.HectorBahamonde.com}.
            \item[] {\large\color{red}\faCamera}\; to check updates on this project. 
        \end{itemize}
\end{frame}


%%%%%%%%%%%%%%%%%%%%%%%%%%%%%%%%
% APPENDIX
%%%%%%%%%%%%%%%%%%%%%%%%%%%%%%%%

\section{Appendix}

\miniframeson
\begin{frame}[plain]{\hypertarget{sum:table:slide}{{\tiny Summary Stats}}}
\center
Test 
\end{frame}


\section{Bibliography}


\begin{frame}[c]
\printbibliography
\end{frame} 


\end{document}




                                                                                                                                                                               
